\subsection*{eth\_test}

\paragraph{v1.0.0 \textit{eth\_test}}
	\begin{itemize}
		\item Comunicazione http configurata con api \texttt{netconn.h}
		\item Aggiunto ritardo per acquisizione ip con dhcp
	\end{itemize}
\paragraph{v1.0.1 \textit{eth\_test}}
	\begin{itemize}
		\item Aggiunta di funzioni per inviare richieste e gestire la risposta
		\item Implementazione del metodo POST nelle richieste
		\item Introduzione delle strutture per il database locale
		\item Formato testuale nel pacchetto in entrata (problemi con scanf e uint8)
	\end{itemize}
\paragraph{v1.0.2 \textit{eth\_test}}
	\begin{itemize}
		\item Aggiunta delle strutture \texttt{shadow\_} per gestire i dati durante il fetch
		\item Chiusura della connessione dopo ogni trasmissione (limitazione di netconn)
		\item Separazione in thread diverse (coordinate da semafori) per distribuire la memoria
	\end{itemize}
\paragraph{v1.0.3 \textit{eth\_test}}
	\begin{itemize}
		\item Migrazione alla api \texttt{socket.h} per poter trasmettere pi\`{u} pacchetti nella stessa connessione
		\item Aggiunto calcolo del checksum 
		\item Ritrasmissione dei pacchetti su errore (checksum)
	\end{itemize}
\paragraph{v1.1.0 \textit{eth\_test}}
	\begin{itemize}
		\item Aggiunta del supporto a mbedTLS, importato da STM32Cube
		\item Inizia il passaggio a HTTPS (http criptato)
		\item Errori nella procedura di handshake con il server
	\end{itemize}
\paragraph{v1.0.4 \textit{eth\_test}}
	\begin{itemize}
		\item Aggiunta del supporto alle ciphersuite con RSA per una migliore compatibilit\`{a}
    \item Ridimensionata il numero di \texttt{MPI} per gestire le chiavi \texttt{RSA}
		\item Risolti problemi sull'handshake, da installare il certificato sul dispositivo
	\end{itemize}
\paragraph{v1.0.5 \textit{eth\_test}}
	\begin{itemize}
		\item Installato il certificato in una sorgente a parte (\texttt{srv\_cert.c})
		\item La connessione si apre e si chiude correttamente
	\end{itemize}

\subsection*{ble\_merge}

\paragraph{v1.0.0 \textit{bl\_merge}}
	\begin{itemize}
		\item Porting da sensor-demo per stm32f401
		\item Adattati i nomi dei pin
	\end{itemize}
\paragraph{v1.0.1 \textit{bl\_merge}}
	\begin{itemize}
    \item Risolti i problemi di spi configurando il pin \texttt{SPI\_IRQ} del BNRG come \texttt{EXTI (EXTernal Interrupt)}
		\item Impostazione come client ble (originariamente server)
	\end{itemize}
\paragraph{v1.1.0 \textit{bl\_merge}}
	\begin{itemize}
		\item Comunicazione con server ble funzionante
		\item Definizione della data come struct
	\end{itemize}
\paragraph{v1.1.1 \textit{bl\_merge}}
	\begin{itemize}
		\item Aggiunto supporto per connettersi a pi\`{u} dispositivi (array contenente tutti gli indirizzi)
		\item Aggiunta del timeout per la connessione
	\end{itemize}

\textbf{unione tra i progetti \textit{eth\_test} e \textit{bl\_merge} nel nuovo progetto \textit{shop429}}

\subsection*{shop429}

\paragraph{v1.2.0 \textit{shop429}}
	\begin{itemize}
		\item Merge dei progetti su eth e ble
		\item Particolare attenzione alle librerie come \texttt{Inc/stm32f4xx\_it} per il corretto funzionamento del ble
	\end{itemize}
\paragraph{v1.2.1 \textit{shop429}}
	\begin{itemize}
		\item Implementazione della macchina a stati anche per la connettivit\`{a} web (\texttt{ETH\_Process()})
		\item Funzione richiamata ad ogni passaggio, necessario l'uso di variabili statiche
		\item Riunione in singola thread del processo di eth rimuovendo i semafori
		\item La versione del database locale viene aggiornata soltanto dopo il fetch completo della tabella
	\end{itemize}
\paragraph{v1.2.2 \textit{shop429}}
	\begin{itemize}
		\item Risolti problemi della macchina a stati ble aggiornando lo stato in base a quello vecchio quando ricominciano le vetrine
		\item Implementazione della scrittura su server (eth) in formato testuale
	\end{itemize}
\paragraph{v1.2.3 \textit{shop429}}
	\begin{itemize}
		\item Uso parziale della variabile locale di stato del negozio
    \item \texttt{PRINTF} configurato per il debug (\texttt{SWV} console)
		\item Ora viene gestita la disconnessione bluetooth anche dall'altro dispositivo
	\end{itemize}
\paragraph{v1.2.4 \textit{shop429}}
	\begin{itemize}
		\item Uso del formato binario nelle connessioni al server web
		\item Inclusa l'autenticazione al server nell'header http
		\item Connessione con database reale sul server (sorgenti php incluse nel progetto)
	\end{itemize}
\paragraph{v1.2.5 \textit{shop429}}
  \begin{itemize}
  	\item Aggiustata la pagina per controllare il database del singolo negozio
    \item Implementate views per rendere accessibile ad ogni utente solo i dati di sua competenza
    \item Aggiustata la funzione \texttt{Keep-alive} per connessioni al server quando altri utenti sono connessi
    \item Il controllo di versione tiene conto anche dell'overflow su un byte
  \end{itemize}
\paragraph{v1.3.0 \textit{shop429}}
  \begin{itemize}
  	\item Implementazione della eeprom simulata per salvare le tabelle
    \item Ad ogni boot vengono ripristinati i dati salvati in rom
  \end{itemize}
\paragraph{v1.3.1 \textit{shop429}}
  \begin{itemize}
  	\item Inizio implementazione del download del firmware via ETH, problemi con pacchetti grandi
    \item Rimosso limite \texttt{SSL\_MAX\_CONTENT\_LEN}, la connessione va in timeout durante il flash
  \end{itemize}
\paragraph{v1.3.2 \textit{shop429}}
  \begin{itemize}
    \item Utilizzo di richieste separate al server per ottenere le varie sezioni del file
  \end{itemize}
\paragraph{v1.3.3 \textit{shop429}}
  \begin{itemize}
    \item Implementazione del trasferimento come \textit{chunked} per sfruttare le potenzialit\`a dell'http (singola richiesta)
  	\item Aggiunto controllo di checksum per il firmware ricevuto
  \end{itemize}
\paragraph{v1.3.4 \textit{shop429}}
  \begin{itemize}
  	\item Migliorata la gestione con costanti della flash
    \item Corretta l'interpretazione della codifica chunked per resistere con dei firmware reali
  \end{itemize}
\paragraph{v1.3.5 \textit{shop429}}
  \begin{itemize}
  	\item Modificato il linker script e l'offset della vector table per consentire l'esecuzione da bootloader
    \item Configurato il reset quando il pacchetto di update \`e stato ricevuto
    \item Implementato il download del firmware per le vetrine
  \end{itemize}
\paragraph{v1.3.6 \textit{shop429}}
  \begin{itemize}
  	\item Impostati valori dello stato del negozio per forzare il download e/o l'installazione di un firmware
    \item Implementato il servizio per caricare il firmware sulle vetrine attraverso il bluetooth
  \end{itemize}
